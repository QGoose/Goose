% easychair.tex,v 3.5 2017/03/15

\documentclass{easychair}
\usepackage{amsfonts}
%\documentclass[EPiC]{easychair}
%\documentclass[EPiCempty]{easychair}
%\documentclass[debug]{easychair}
%\documentclass[verbose]{easychair}
%\documentclass[notimes]{easychair}
%\documentclass[withtimes]{easychair}
%\documentclass[a4paper]{easychair}
%\documentclass[letterpaper]{easychair}

\usepackage{listings}
\usepackage[skip=0pt]{caption}
\usepackage[T1]{fontenc}
\usepackage[english]{babel}
\usepackage{doc}
\usepackage{soul}
\usepackage{minted}
\usepackage{tikz}

\newmintinline{OCaml}{}

\title{Goose: An OCaml environnement to simulate quantum programs}

\author{
  Arthur Correnson\inst{1}
\and
  Chris McNally\inst{2}
\and
  Youssef Moawad\inst{3}
\and
  Denis Carnier\inst{4}
}

\institute{
  ENS de Rennes,
  \email{arthur.correnson@ens-rennes.fr}\\
\and
  MIT
\and
  University of Glasgow
\and
  DistriNET
}

\authorrunning{A. Correnson, F. Bobot}

\titlerunning{Quantum programs in OCaml}

\begin{document}

\maketitle

\begin{abstract}
  A quantum Abstract
\end{abstract}

%------------------------------------------------------------------------------
\section{Introduction}

A quantum intro

\section{Conclusion and Future Work}

Quantum future worlds

\bibliographystyle{plain}
\bibliography{biblio}

%------------------------------------------------------------------------------
% Index
%\printindex

%------------------------------------------------------------------------------
\end{document}
